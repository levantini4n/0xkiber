\documentclass[11pt]{article}

% Use the same packages and settings as main.tex
\usepackage[utf8]{inputenc}
\usepackage[T1]{fontenc}
\usepackage{lmodern}
\usepackage[paperwidth=330mm,paperheight=450mm,margin=0.3in]{geometry}
\usepackage{xcolor}
\usepackage{colortbl}
\usepackage{hyperref}
\usepackage{graphicx}
\usepackage{enumitem}
\usepackage{booktabs}
\usepackage{listings}
\usepackage[breakable,skins,theorems]{tcolorbox}
\usepackage{titlesec}
\usepackage{fancyhdr}
\usepackage{multirow}
\usepackage{wrapfig}
\usepackage{microtype}
\usepackage{pdfpages}
\usepackage{bookmark}
\usepackage{caption}

% Add this package to help with PNG files
\usepackage{epstopdf}

% Color definitions
\definecolor{codebackground}{rgb}{0.95,0.95,0.95}
\definecolor{commandcolor}{rgb}{0.8,0.0,0.0}
\definecolor{outputcolor}{rgb}{0.0,0.0,0.6}
\definecolor{commentcolor}{rgb}{0.4,0.8,0.4}
\definecolor{sectioncolor}{rgb}{0.0,0.3,0.5}
\definecolor{subsectioncolor}{rgb}{0.0,0.4,0.4}
\definecolor{subsubsectioncolor}{rgb}{0.0,0.5,0.7}
\definecolor{warningcolor}{rgb}{0.9,0.5,0.3}
\definecolor{tiphighlight}{rgb}{0.95,0.95,0.7}
\definecolor{kalibackground}{rgb}{0.15,0.15,0.15}
\definecolor{kalitext}{rgb}{0.4,0.7,1.0}
\definecolor{kaliprompt}{rgb}{0.2,0.8,0.8}
\definecolor{kalicommand}{rgb}{0.4,0.7,1.0}
\definecolor{kalioutput}{rgb}{0.4,0.7,1.0}
\definecolor{kaliurl}{rgb}{0.4,0.7,1.0}
\definecolor{kaliheader}{rgb}{0.4,0.7,1.0}

% Custom environments
\newenvironment{commandbox}[1][]{
    \begin{tcolorbox}[
        colback=kalibackground,
        colframe=commandcolor,
        fonttitle=\bfseries\color{white},
        title=#1,
        breakable=true
    ]
}{
    \end{tcolorbox}
}

% Code listing settings
\lstset{
    backgroundcolor=\color{kalibackground},
    basicstyle=\footnotesize\ttfamily\color{warningcolor},
    breakatwhitespace=false,
    breaklines=true,
    captionpos=b,
    commentstyle=\color{commentcolor},
    keepspaces=true,
    keywordstyle=\color{kalitext},
    showspaces=false,
    showstringspaces=false,
    showtabs=false,
    tabsize=2,
    moredelim=[il][\color{commentcolor}]{\$\ },
    stringstyle=\color{kalitext}
}

\lstdefinestyle{bash}{
    morecomment=[l][\color{commentcolor}]{\#}
}

% Change subsubsection numbering to Roman numerals while keeping the full hierarchy
\renewcommand{\thesubsubsection}{\thesubsection.\Roman{subsubsection}}

% Section title formatting to match main.tex
\titleformat{\section}
{\color{sectioncolor}\normalfont\Large\bfseries}
{\color{sectioncolor}\thesection}{1em}{}

\titleformat{\subsection}
{\color{subsectioncolor}\normalfont\huge\bfseries}
{\color{subsectioncolor}\thesubsection}{1em}{}

\titleformat{\subsubsection}
{\color{subsubsectioncolor}\normalfont\Large\bfseries}
{\color{subsubsectioncolor}\thesubsubsection}{1em}{}

% Remove page numbers
\pagestyle{empty}

% Set a much larger page size to match aggressive-scan.tex
\special{papersize=8.5in,14in}
\setlength{\pdfpageheight}{14in}
\setlength{\pdfpagewidth}{8.5in}

\begin{document}

\setcounter{section}{2}
\setcounter{subsection}{3}

\subsection{\huge \color{subsectioncolor}TCP/UDP Host Discovery}

\subsubsection{\LARGE \color{subsubsectioncolor}TCP SYN Scan (-PS)}

{\Large Nmap will send TCP SYN packets and won't complete the TCP 3-way handshake even if the port is open, as shown in the figure below the terminal's output.}

\begin{commandbox}[TCP SYN Scan (-PS)]
\begin{lstlisting}[language=bash, style=bash, basicstyle=\normalsize\ttfamily\color{warningcolor}]
$ sudo nmap -PS -sn 10.10.68.220/24
\end{lstlisting}

\begin{lstlisting}[basicstyle=\normalsize\ttfamily\color{kalitext}]
Starting Nmap 7.92 ( https://nmap.org ) at 2021-09-02 13:45 EEST
Nmap scan report for 10.10.68.52
Host is up (0.10s latency).
Nmap scan report for 10.10.68.121
Host is up (0.16s latency).
Nmap scan report for 10.10.68.125
Host is up (0.089s latency).
Nmap scan report for 10.10.68.134
Host is up (0.13s latency).
Nmap scan report for 10.10.68.220
Host is up (0.11s latency).
Nmap done: 256 IP addresses (5 hosts up) scanned in 17.38 seconds
\end{lstlisting}
\end{commandbox}

\begin{center}
\includegraphics[width=0.7\textwidth]{SYNPack.png}
\captionof{figure}{SYN Packet sent for Host Discovery, the user has root privileges.}
\label{fig:SYN Packet}
\end{center}

\end{document} 